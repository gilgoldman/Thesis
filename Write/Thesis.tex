% arara: pdflatex: { synctex: yes }
% arara: makeindex: { style: ctuthesis }
% arara: bibtex

% The class takes all the key=value arguments that \ctusetup does,
% and a couple more: draft and oneside
\documentclass[twoside]{ctuthesis}
\usepackage{graphicx}
\usepackage{listings}


\ctusetup{
%	preprint = \ctuverlog,
	mainlanguage = english,
%	titlelanguage = czech,
%	mainlanguage = czech,
	otherlanguages = {czech},
	title-czech = {Title in Czech},
	title-english = {Alcohol content measurement within the fermentation process},
	subtitle-czech = {Subtitle in Czech},
	subtitle-english = {\textit{Low-cost automatic ethanol measurement as indication of fermentation progress}},
	doctype = B,
	faculty = F3,
	department-czech = {Department in Czech},
	department-english = {Department of Control Engineering},
	author = {Gil Goldman},
	supervisor = {Prof. Jiři Novák},
	supervisor-address = {Ústav X, \\ Uliční 5, \\ Praha 99},
	fieldofstudy-english = {Robotics and Cybernetics},
	subfieldofstudy-english = {Control Engineering},
	fieldofstudy-czech = {Robotika a Kybernetika},
	subfieldofstudy-czech = {Subfield in Czech},
	keywords-czech = {Keywords in Czech},
	keywords-english = {Fermentation, Ethanol, Automation, Measurement},
	day = 20,
	month = 4,
	year = 2018,
	specification-file = {ctutest-zadani.pdf},
%	front-specification = true,
%	front-list-of-figures = false,
%	front-list-of-tables = false,
%	monochrome = true,
%	layout-short = true,
}

\ctuprocess

\addto\ctucaptionsczech{%
	\def\supervisorname{Vedoucí}%
	\def\subfieldofstudyname{Studijní program}%
}

\ctutemplateset{maketitle twocolumn default}{
	\begin{twocolumnfrontmatterpage}
		\ctutemplate{twocolumn.thanks}
		\ctutemplate{twocolumn.declaration}
		\ctutemplate{twocolumn.abstract.in.titlelanguage}
		\ctutemplate{twocolumn.abstract.in.secondlanguage}
		\ctutemplate{twocolumn.tableofcontents}
		\ctutemplate{twocolumn.listoffigures}
	\end{twocolumnfrontmatterpage}
}

% Theorem declarations, this is the reasonable default, anybody can do what they wish.
% If you prefer theorems in italics rather than slanted, use \theoremstyle{plainit}

\theoremstyle{plain}
\newtheorem{theorem}{Theorem}[chapter]
\newtheorem{corollary}[theorem]{Corollary}
\newtheorem{lemma}[theorem]{Lemma}
\newtheorem{proposition}[theorem]{Proposition}

\theoremstyle{definition}
\newtheorem{definition}[theorem]{Definition}
\newtheorem{example}[theorem]{Example}
\newtheorem{conjecture}[theorem]{Conjecture}

\theoremstyle{note}
\newtheorem*{remark*}{Remark}
\newtheorem{remark}[theorem]{Remark}

%\setlength{\parskip}{5ex plus 0.2ex minus 0.2ex}

% Abstract in Czech
\begin{abstract-czech}
Abstract in Czech
\end{abstract-czech}

% Abstract in English
\begin{abstract-english}
 In this dissertation I will survey the current methods of measuring the progression of microbrewed beer fermentation processes. In addition, I will implement an automatic solution to monitor the gradual accumulation of ethanol in the process of beer fermentation.
\end{abstract-english}

% Acknowledgements / Podekovani
\begin{thanks}
The Author would like to thank Professor Jiri Novak for his thorough guidance and tutelage - it has been an honor working on this topic under his supervision. %\emph{alma mater}.
\end{thanks}

% Declaration / Prohlaseni
\begin{declaration}

Prohlašuji, že jsem předloženou práci vypracoval samostatně, a že jsem uvedl veškerou použitou literaturu.\\
V Praze, \ctufield{day}.~\monthinlanguage{title}~\ctufield{year}\\

\vspace{5mm} % a small space between both paragraphs

I declare that this work is all my own work and I have cited all the sources I have
used in the bibliography.\\

In Prague, \ctufield{day}.~\monthinlanguage{title}~\ctufield{year}

\end{declaration}

% Only for testing purposes
\listfiles
\usepackage[pagewise]{lineno}
\usepackage{lipsum,blindtext}
\usepackage{mathrsfs} % provides \mathscr used in the ridiculous examples

\begin{document}

\maketitle

\section{Introduction}

In this dissertation I will provide a brief overview of the way beer is made by home brewers today - I will follow the process from grains to beer and observe a few issues with the common way this process is carried out today in most microbreweries.\\
Then, I will explore the various ways by which ethanol is currently being measured in home-grade brewing processes, also known as micro-brewing, and evaluate the advantages and disadvantages of each method. I will then discuss why and how the method suggested by this paper is the most suitable.\\
I will then establish and justify an automatic, low cost solution to the issue of accurate and safe ethanol measurement, followed by rigorous testing of the proposed solution.

\chapter{How beer is brewed}

In this chapter, I will elaborate on how beer is being made, and discuss the most common method of micro-brewing.
From this explanation I may easily identify the weakness in the process and the justification for this thesis.

\section{Overview of Beer Brewing}
In general, micro-brewing can be seen as a fairly simple process, composed of five significant stages: \textit{Malting}, \textit{Mashing}, \textit{Worting}, \textit{Fermentation}, and \textit{Packing}.\\
A short flow chart below illustrates the general principles of the brewing process before I dive deeply into the various steps:

\begin{figure}[H]
\centering
\includegraphics[scale = 0.39]{BeerMakingFlowChart}
\caption{Flow chart of the Brewing process}
\end{figure}


\section{Malting}
The first step in beer brewing is called \textit{Malting}, and it is the name given to the process of preparing the grain to be \textit{mashed}. The most common grains to be malted are Barley (\textit{Hordeum Vulgare}) and Wheat (\textit{Triticum aestivum}). Sorghum (\textit{Sorghum Vulgare}) is also rather common, but mostly in its indigenous continent of Africa. Some varieties of rye, oats, and millets are also used, but to a significantly lesser extent \cite{Brewing_Science}.\\
The main goal of malting is to germinate the grains used in the brewing process, breaking the $\alpha-amylase$ and $\beta-amylase$ enzymes out of the \textit{amylose homologous} series in the grains. These Enzymes will later be used to break the grain starches into various saccharides .\\
Malting is most commonly done by steeping the grains in and out of water until they reach about 45$\%$ moisture content, and then maintaining that high moisture content via bursts of highly humidified air. The germination process is stopped by $Kilning$ - blowing hot, dry air through the grains to reduce their inherent moisture content down to 5$\%$ \cite{Malting_Brewing}. Various flavors and colors can be developed by changing the duration and temperature of the Kilning process. \cite{Malting} \\
The final step of the Malting process is done as close to brewing as feasible, and involves cracking and grinding the grains to allow easy extraction of the starches during the Mashing stage. The end product of this process is called $Malt$.\\
As this process is relatively expensive and mechanically demanding, the majority of micro-breweries buy malt rather than produce it.

\section{Mashing}
The process of $mashing$ involves steeping, or cooking, the malt in water at specific temperatures to allow the enzymes developed during the malting process to take effect.\\
The $\alpha-amylase$ enzyme's main function is to break the large, complex, insoluble starches in the grain into smaller, simpler, and soluble starches. The $\beta-amylase$ converts the water soluble starches into usable types of sugar, such as the monosaccharide glucose, the disaccharide maltose, the trisaccharide maltotriose, and various other, more complex, sugars. Most notable among these is the disaccharide $Maltose$, which is the main sugar processed by the $\alpha$ and $\beta$-amylase enzymes.\\
There are various mashing methods, most notable are \textit{infusion mashing} and \textit{decoction mashing}.\\
Infusion mashing involves steeping the grains in water, slowly increasing the temperature of the water, and stopping at pre-designated stops - the goal of which is to encourage the enzymes to break the starches into sugars without denaturing them. This is the easier alternative, requiring nothing more than a source of heat, a thermometer, and a timer. \\
Decoction mashing involves removing set amounts of grain at set times from the brewing mash, boiling them in a separate vessal to encourage a Mallard reaction, and reintroducing the now hotter grains into the mash in order to increase its overall temperature. This method is far more complex than infusion mashing, yet produces greater quantities of maltose, better calculated fermentability rates, as well as more noticeable flavors and aromas \cite{Brewing_Science}.\\
A review of the summary of the summarized table below, comparing the results of various methods, mashes, syrups, will reinforce the above statement.

\begin{figure}[H]
	\centering
	\includegraphics[width = \textwidth]{MashingTable}
\begin{table}[H]
	\caption{The carbohydrate compositions of two worts and several syrups prepared from starches ($\%$)\cite{Brewing_Science}}
\end{table}
\end{figure}

\section{Worting}
In favor of clarity, in the scope of this work I will define a step of the process which I will name "Worting". To the experienced, it is a combination of Lautering and secondary boiling.\\
After the mashing process is complete, I will begin Worting the mash. Worting involves two main steps:\\
The first step consists of separating the grains from the mash. This is commonly achieved in large breweries by filtering the grains from the mash as it is being transferred into a secondary pot for the Worting process. Microbreweries sometimes preform the same action, however it is common to use only one pot, resulting in use of specialized bags or sieves to hold the grains during the mashing process, and removing them before Worting. At this stage, Hops (\textit{Humulus lupulus}) are added into the brewing mixture in order to add flavor and texture, as well as control the bitterness of the final Product.\cite{Hops}\\
The second step of the process involves boiling the now grain-free mash to eliminate bacteria and sterilize the mixture - resulting in a sugary grain juice named Wort.

\section{Fermentation}
After the Worting process is complete, the Wort is chilled to a predetermined temperature range which depends on the type of beer being brewed. The chilled wort is transferred into a new air sealed container where yeast of the genus \textit{Saccharomyces} are added to it at a common rate of $15-20 x 10^6$ cells per [m$L^{-1}$] - a process called Pitching. The yeast which are added to the Wort will consume the abundant sugars and convert them into ethanol and higher alcohols, all the while producing $CO_2$ as a byproduct. The yeast's action will over time transform the sugary grain juice into what is recognized as beer.\\
During this relatively long process which lasts from a few days up to several weeks, the most common way of measuring the progress of the fermentation process is via extracting a sample of wort by hand and measuring the liquid's specific gravity - the ratio between the density of the liquid and that of water measured at 4[$C^\circ$] - by dipping a hydrometer in the sample.\\
As fermenting beer is sensitive to contamination and oxidization, this method of measurement, by far the most common one, is far from optimal. \cite{Biochemistry}

\section{Packing}
After the yeast have finished their work, the final step of the process is to package the beer in sterilized bottles or cans. In most microbreweries, the bottling is conducted by utilizing specialized tools to deliver the beer from the fermentation tanks to the bottles while limiting contact with the surrounding air as much as possible - contact with air at this step exposes the beer to severe contamination risk,  which endangers the products safety, and oxidation, which leads to considerable worsening of flavor and taste.\\
In most microbreweries it is common to add additional saccharides - most commonly white sugar or glucose syrup - to the bottled beer in order to stimulate additional fermentation after capping the bottles, ensuring sufficient carbonization of the final product.

\section{Reasons for this thesis}
As I now know, the nature of the fermentation process as it is carried out in most microbreweries means it cannot be continuously or accurately measured, as every measurement of the fermentation process endangers the final products safety and taste.\\
In addition, accurate measurement of the fermentation process would allow early bottling of the beer, eliminating the need to add sugar or saccharides to stimulate sufficient carbonization and assuring a healthier, tastier, and safer product.

\chapter{Fermentation Measurement Methods}
In this chapter, I will discuss the various methods used by different institutes and businesses to monitor the progression of Wort fermentation processes - most commonly by measuring in various ways its Ethanol content.\\
I will investigate five methods of Ethanol measurement:\\

\begin{itemize}
	\item Densitometry
	\item Near and Mid Infra-red Spectroscopy
	\item Gas Chromatography
	\item Hydrometry
\end{itemize}

In the following chapter, I will discuss the methodology of these five alternatives, as well as examine their use cases and feasibility for use in micro-brewing and large breweries alike.\\

\newpage

\section{Densitometry}
The first method of measuring ethanol in beer fermentation processes I will explore is utilizing a digital density meter.\\
As fermentation progresses, the yeast pitched into the Wort earlier will convert the saccharides into ethanol, higher alcohols, and carbon dioxide. This process eliminates the Worts sweetness, revealing the bitterness given by the hops, and transforming the Wort into beer. \\
While this process is not fully explored yet \cite{Brewing_Science}, it has a measurable effect on the Worts density, an effect which, while not linear, is pretty well understood and charted.\\
As the correlation between Worts density and alcohol content is well explored, it is possible to identify current alcohol content in a sample by identifying its density. Since finding out a liquids density is rather challenging, it is common to use a digital density meter\cite{Ethanol_Measurement}.\\

\begin{figure}[H]
	\centering
	\includegraphics[scale = 0.6]{sg-ultra-max-digital-densitymeter-d}
	\caption{A SG digital density-meter}
\end{figure}

A digital density meter works by extracting a small sample of liquid, and injecting it into an oscillating U shaped tube. The tube is then piezoelectrically or electromagnetically excited into un-damped oscillation, vibrating two tubes - one with a sample, the other with a reference material. As the oscillating volume is known, it is possible to deduce its density from the period in which it oscillates based on the following relation:
\begin{equation}
	\tau = 2\pi \sqrt{\frac{\rho_sV_c+m_c}{\mathcal{K}}}
\end{equation}
Where $\rho_s$ is the density of the liquid to be discovered, $V_c$ is the internal volume of the u-tube, $m_c$ is the mass of an empty u-tube, and $\mathcal{K}$ is a manufacturer defined constant.\\

This method is mostly chosen due to its flexibility and reliability - substances will oscillate in different frequencies directly affected by their respective densities, and overfilling the device will not impact the measurement results\cite{Density_Measurement}.
\begin{figure}[H]
	\centering
	\includegraphics[scale = 0.45]{u-tube-density-meter}
	\caption{An example of a digital density meter working principle}
\end{figure}

While common in many industrial fields, a digital density meter is an expensive device. While it may be suitable for companies or research laboratories, it is not so fitting for private individuals.

\section{Near and Mid Infra-red Spectroscopy}
Near and Mid Infra-red Spectroscopy, respectively NIRS or MIRS, are twin methods which may be used for measuring ethanol content in liquids. To avoid repetitivity, I will explore how NIRS work, and detail the major differences between the methods.\\
NIRS is a spectroscopic method which utilizes electromagnetic radiation(EMR) from the near infrared region, typified by wavelengths of 700-1100 $[nm]$\cite{NIR_Spectroscopy_Ethanol}. In very broad strokes, Spectroscopy may be defined as studying the way in which different molecules react to EMR, and can be seen as the implementation of Beer-Lamberts law, which describes the relation between the attenuation of light to the properties of the material through which the light is traveling.\\
Since specific molecules diffract specific wavelengths, a samples composition may be understood and analyzed by studying which wavelengths of EMR are absorbed by it.

\begin{figure}[H]
	\centering
	\includegraphics[scale = 0.75]{spectrometer}
	\caption{NIRS DS2500 Analyzer by Metrohm NIRSystems}
\end{figure}

A spectrometer can be similarly defined as an instrument which illuminates a sample material with EMR of various wavelengths and measures the diffraction of electromagnetic radiation caused by the sample material. Most spectrometers work by having a light source shine light through a prism or other light diffracting objects, such as specialized grates. The diffracted light is then filtered via a movable slit, allowing to select a specific wavelength range, which is then shined at a photo-diode or photo-transistor through the tested sample. The measured current generated by the photo-diode is then converted into a useful reading.

\begin{figure}[H]
	\centering
	\includegraphics[scale = 0.7]{spectrometer_scheme}
	\caption{Illustration of how a spectrometer works}
\end{figure}

In our relevant case, a good correlation has been found between the presence of Ethanol molecules and the intensity of backscattered light at 905 $[nm]$  \cite{NIR_Spectroscopy_Ethanol}, allowing us to identify it in sample compounds.\\
MIR and NIR, being two different approaches to the same result, are often used in conjunction for better results - NIr having greater sample penetrability and MIR suffering from less noise.
Until rather recently, NIRS instruments required sanitized environments, highly trained operators, and were generally quite large and bulky - making them more suitable for lab work rather than field work\cite{NIR_For_Spices}.\\ While in recent years more modern devices are being prototyped which will be smaller and can operate in a wider range of environments, using a spectrometer still requires specific training and experience, and a spectrometer is still an extremely expensive device.

\section{Gas Chromatography}
Gas Chromatography is a rather less common method of analysing beer, and is mostly used by research institutes and tax authorities, as it provides rather consistent reproducibility and works well with small samples.\\
Chromatography is an umbrella term for several methods following a common principle - different materials have different adsorption rates. Adsorption is the term given to the tendency of various atoms and molecules to stick to certain materials in differing rates. The adhesive properties of materials are determined experimentally for specific material-molecule pairs - for example, a 2004 study established that the saturation coverage of Ethanol on Silicon is 0.42 $\pm$ 0.10.\cite{Ehtnaol_adsorption,Gas_Chromatography_beer}\\

\begin{figure}[H]
	\centering
	\includegraphics[scale = 0.4]{refurbished-gas_chromatograph}
	\caption{Gas Cromatograph (picture by BVK Technology Services)}
\end{figure}

When performing a gas chromatography test, a sample of beer is retrieved and injected into the gas chromatograph via a mechanical syringe. \\
The sample is evaporated and immediately mixed with an eluant - a neutral, non-reactive carrier gas, in most cases Helium. The Helium assists the gaseous mixture in travelling through the $column$, a thin metal or glass tube which houses a liquid with a high boiling point. \\
As the gaseous mixture travels through the heated tubing, it separates into its constituent parts. The samples components travel through the tubing in different velocities, until they are expelled through a detector at the end of the tubing, which varies by maker.

\begin{figure}[H]
	\centering
	\includegraphics[width = \textwidth]{Gas_Chromatograph_PDF}
	\caption{Gas Cromatograph flow chart}
\end{figure}

Gas chromatography is highly complex and very expensive method which has little to offer for micro-breweries. While it is very precise and has high reproducibility potential, these are traits which are less important to the average micro-brewer. Therefore, it is almost never implemented within this context, except perhaps by those most pedantic about accuracy of results.

\newpage

\section{Hydrometry}
Hydrometry is the most common of all ethanol measurement techniques, as it is the cheapest and easiest to understand and utilize.\\
In the next chapter I will delve deeply into how a hydrometer works and why it was chosen as the cornerstone method of the physical implementation of this thesis.


\chapter{Solution proposition}
In this chapter I will discuss how hydrometry work and why it is the most common method for evaluating the amount of ethanol in fermenting beer, examine its greatest issue, and propose a solution for this risk.

\section{How Hydrometry work}
Hydrometry is simple, easy, and cheap to perform, and provides a sufficiently accurate measurement of ethanol contents.\\
Hydrometry can be viewed as an application of Archimedes principle: "Any object immersed in fluid, partially or wholly, is acted upon by a buoyant force equal to the weight of the fluid displaced by it". A hydrometer is a marked instrument, commonly made from glass and weighed by lead\cite{Ethanol_Measurement}, weighted so as to achieve neutral buoyancy, with the point on the hydrometer at water level while it is in buoyant equilibrium marked as the instruments baseline. The baseline is marked as $1.00$, to denote that there is no difference between the fluids density and that of water. Due to historical reasons, in Britain it is commonly marked up by three orders of magnitude, marking the baseline as $1000$\cite{Brewing_Science} instead of $1.00$\\
As the hydrometer floats around a certain equilibrium point in water, and its precise weight and volume is known, it is possible to deduce the sample fluids specific gravity - its density compared to that of water - based on the difference between the height of the current equilibrium point relative to that of water. In simpler terms, the denser the fluid, the higher the hydrometer will float in the fluid due to a stronger buoyant force acting on it.\\
The instrument is notched or marked in various distances to simplify reading its results, each notch indicating a different specific gravity. It is also calibrated around a specific temperature, in most cases 15$[C^{\circ}]$\cite{Ethanol_Measurement}. As temperature has a noticeable effect on density, and as a result on specific gravity(SG), most charts detailing correlation between SG and hydrometer readings also correct for temperature differences.\\


\begin{figure}[H]
	\centering
	\includegraphics[scale=0.7]{hydrometer}
	\caption{Hydrometer operating principle \cite{Hydrometer_Pic}}
\end{figure}

In order to use a hydrometer, a sample of the Wort is retrieved after it has cooled to the desired temperature, but before the yeast has been pitched (see section 1.5). This reading will be designated as the Original Gravity(OG) of the batch being brewed, and all changes in the Worts SG due to an increase in ethanol content will be compared to this. While it is possible to estimate from the OG the finite products alcohol content, it is rarely accurate.

\section{The risk of Hydrometry}
To ensure accuracy, and to know when to proceed to the next stage of the brewing process, at least several measurements by hydrometer are required. However, every measurement involves retrieving a sample of fermenting Wort in order to measure its specific gravity.\\
Several solutions have been implemented to minimize exposure to contaminants and oxygen, which, at this stage of the process, can cause staling of the beer and a severe damage to its taste and clarity. However, most of them are mechanical, and none of them has completely solved the issue.\\
Today, each micro-brewer must find their own point of balance between accuracy and risk, with each measurement increasing the accuracy of the process but endangering the final product.



\section{Proposed solution}
Therefore, I propose eliminating the need to retrieve a sample of the fermenting beer altogether, and disposing of the last major risk factor in beer brewing.\\
The proposed solution will ensure a continuous - or, more accurately, discrete over very short intervals - measurement of the fermenting Worts SG, as well as calculate the current alcohol content while factoring in the environmental temperature. The device proposed will be integrated as a part of the closed fermenting Wort system, thus eliminating the danger of oxidation or contamination. Furthermore, it will allow wireless access to the results, so as to provide greater transparency of the brewing process.\\
The device will be a Raspberry pi zero w attached to an HC-sr04 ultrasonic distance sensor and a DHT-11 temperature sensor - the exact architecture will be discussed at length in Chapter 4 of this work. The prototype designed and built for this thesis will be powered via cable, however future version may be battery operated. In order to ease integration and reduce costs, the device will utilize the hydrometer already present at every micro-brewery to complete the system. Should a micro-brewery not have a hydrometer, one may be supplied.\\

\begin{figure}[H]
	\centering
	\includegraphics[scale=0.4]{Housing}
	\caption{General solution description}
\end{figure}

The final result will be a low cost, accurate, and reliable solution to provide insight into the process while eliminating the most common danger factors of traditional methods. In the next chapter I will elaborate on both the systems software and hardware architecture, as well as its implementation in the field.


\pagebreak

\begingroup
\renewcommand{\cleardoublepage}{}
\renewcommand{\clearpage}{}
\chapter{Technical Implementation}
\endgroup

The devices design can be broken down into two parts: physical architecture and software architecture.\\
In this chapter, both of the architectures will be described, and their respective implementation discussed. The design schematics and pictures will be provided where relevant. Each part will be described, analysed, and justified where relevant.

\section{Physical Architecture}

The device will be mounted inside a custom housing which will hold all of the relevant components. These components are:

\begin{itemize}
	\item Raspberry Pi Zero W
	\item Two HC-sr04 Ultrasonic sensors
	\begin{itemize}
		\item Two 1,000 [$\Omega$] resistors
		\item Two 2,000 [$\Omega$] resistors
	\end{itemize}
	\item DHT-11 Digital Humidity and Temperature sensor
	\begin{itemize}
		\item 10,000 [$\Omega$] resistor
	\end{itemize}
	\item Mini Breadboard
	\item Hydrometer
	\item Lattice/rice paper attachment
	\item Housing unit
\end{itemize}

With the exception of the lattice/paper attachment and the Housing unit, the estimated cost of parts for the entire project stands at 12.84 $USD$, or 264.61 $CZK$.\\
Next, I will examine and justify each component selection.

\subsection{Raspberry Pi Zero W}
The Raspberry Pi Zero W, henceforth RPi, is a low cost micro-computer, currently priced at 10$\$$ USD. The RPi is small, measuring only 0.065[m] x 0.030[m] x 0.005[m], making it ideal for small device control. It also features a 1[GHz] ARM11 processor, 512[MB] of RAM, and a wireless LAN 2.4 [GHz] 802.11n surface-mount component etched into the board.\\
This device was chosen for its ease of use, availability, versatility, and low cost. It is fairly durable, and easily replaceable.

\begin{figure}[H]
	\centering
	\includegraphics[scale=0.15]{RasPIZeroW}
	\caption{Raspberry Pi Zero W micro-computer \cite{RasPi0W}}
\end{figure}


\subsection{HC-sr04 Ultrasonic Ranging Module}
The HC-sr04 sensor provides a non-contact ultrasonic distance measurement function, at ranges of 2 to 400 [cm] $\pm$0.3[cm] with a measuring angle of 30$^\circ$\\
The sensor works by emitting eight pulses at 40[kHz] and listening for the rebounding pulses. It is then possible to computationally calculate, based on the length of the rebounding signal, the distance between the sensor and the target.\\
The HC-sr04 sensor was chosen for this project due to it's high accuracy, small size, and low price. However, as the sensor's accuracy fluctuations can have a noticeable impact on such sensitive instrumentation, methods to compensate for them will be discussed and tested in Chapter 5.\\

\begin{figure}[H]
	\centering
	\includegraphics[scale=0.5]{HCSR04_sizes}
	\caption{HC-sr04 dimensions and angle accuracy \cite{HC-SR04}}
\end{figure}

As the sensor's output signal via the ECHO pin is rated at 5[V] \cite{HC-SR04} and the Raspberry Pi's GPIO (General Purpose Input Output) is rated at 3.3[V]\cite{RasPi0W}, it would be wise to implement some sort of protection for the board. The most effective way to provide such protection is a voltage divider, with the sensors output connected through it to the board. The values for the derivation of the required resistors can be found as follows:

\begin{gather}\nonumber
	\frac{V_{output}}{V_{in}} = \frac{R_2}{R_1 + R_2}\\\nonumber
	\frac{3.3}{5}= 0.66 = \frac{R_2}{R_1 + R_2}\\\nonumber
	0.66(R_1+R_2) = R2\\\nonumber
	0.66R_1 = 0.34R_2\\
	1.941R_1 = R_2
\end{gather}

Thanks to $(4.1)$, it is now clear that $R_2$ should have approximately twice the resistance of $R_1$. Therefore, in the scope of this device, the two resistors chosen were of $1000\Omega$ and $2000\Omega$, due to their abundance and price.


\subsection{DHT-11 Digital Humidity and Temperature sensor}
The DHT-11 Digital Humidity and Temperature utilizes a capacitive humidity sensor and a thermistor to measure relative ambient humidity with $\pm$5$\%$ accuracy at 25$^\circ$, and temperature with $\pm$2 [C$^\circ$] at 25[C$^\circ$]. The instruments readings are then digitalized via a built in microprocessor which transmits a digital reading from each sensor at 16 bit resolution.

\begin{figure}[H]
	\centering
	\includegraphics[scale=0.5]{DHT11}
	\caption{DHT-11 Digital Humidity and Temperature sensor\cite{HC-SR04}}
\end{figure}

\subsection{Additional Components}
The rest of the components which are needed are generic and not type dependant: a mini breadboard, a hydrometer, several resistors, and a piece of paper. The housing unit can be a recycled bottle even, so while for the purposes of this work I will use a 3d printed one, any can be used in its stead.\\
\paragraph{Mini Breadboard} Will be used to simplify prototyping - in a final product the various components will be soldered to each other, making a breadboard obsolete.
\paragraph{Hydrometer} Is a generic hydrometer - the specific gravity of the liquid will be deduced by the change in the hydrometer height in wort compared to its height in water. Any hydrometer will do.
\paragraph{Resistors} Are used either as pull-up resistors or to form voltage dividers for the various sensors.
\paragraph{Lattice} Or the piece of rice paper will act as a lightweight extension added on top of the hydrometer to ease its detection by the HC-sr04. This attachment will be small and lightweight enough to make its addition to the system negligible.

\section{Software Architecture}
The system will be coded entirely in Python to support modularity and ease of maintenance, supplemented by open-source software with appropriate licenses when necessary to assure high quality. The software architecture can be viewed as composed of three main parts:\\

\begin{itemize}
	\item Raspbian Stretch
	\item Cron linux module
	\item Main software body
	\item Grafana 
	\item InfluxDB
	\item Distance sensor thread
	\item Temperature sensor thread
\end{itemize}

\subsection{Raspbian Stretch}
Raspbian Stretch is a Linux based operating system (OS) adapted for use with ARM processors. The version used in the scope of this work is that of April 2018, the latest in a long line of modifications made to the basic Raspbian system at the time of writing.\\
It was chosen as it is an OS specifically built to work well with Raspberry Pi's and their unique hardware.

\subsection{Cron Linux Daemon}
Cron, named after the Greek word for time, is a Linux Daemon - software which run in the background of an operating system as a process independent from user interaction. Cron, specifically, is a Daemon which assists in scheduling tasks, as it runs automatically once a minute based on the systems clock. It will be used to activate the Main software module automatically as the system boots, as well as following and monitoring additional threads and processes created by the Main module.

\subsection{Main Software module}
The main software body is the 

\section{Implementation: Physical Architecture}





\section{Implementation: Software Architecture}






%\part{My Party}

%\input{ctutest-1}

%\input{ctutest-2}


%\part{Your Party}



%\appendix

%\printindex

\appendix

%\bibliographystyle{amsalpha}
%\bibliography{ctutest}
\begin{thebibliography}{9}
	\bibitem{Brewing_Science}
	\textit{\textbf{Brewing Science and Practice}}\\
	D. E. Briggs ; C. A. Boulton ; P. A. Brooks ; R. Stevens\\
	Woodhead Publishing Limited and CRC Press LLC, 2004

	\bibitem{Brewing}
	\textit{\textbf{Brewing}}\\
	M. J. Lewis ; T. W. Young\\
	Springer Science \& Business Media\\
	\textit{6.12.2012}\\
	ISBN 9780306472749

	\bibitem{Malting_Brewing}
	\textit{\textbf{Malting and Brewing Science: Volume II Hopped Wort and Beer}}\\
	J. S. Hough ; D. E. Briggs ; R. Stevens ; T. W. Young\\
	Springer Science \& Business Media\\
	ISBN 978-1-4613-5727-8

	\bibitem{Malting}
	\textit{“Effect of Malting Temperature and Mashing Methods on Sorghum Wort Composition and Beer Flavour.”} \\
	M. A. Igyor, et al. \\
	\textit{Process Biochemistry}, vol. 36, no. 11, 2001, pp. 1039–1044.\\
	doi:10.1016/s0032-9592(00)00267-3.

	\bibitem{Hops}
	\textit{“125th Anniversary Review: The Role of Hops in Brewing”}\\
	C. Schonberger ; T. Kostelecky\\
	\textit{Journal of the Institute of Brewing}, vol. 117, no. 3, 2001, pp. 259-267.\\
	doi:10.1002/j.2050-0416.2011.tb00471.x

	\bibitem{Biochemistry}
	\textit{\textbf{Biochemistry of Beer Fermentation}}\\
	E. Pires ; T. Branyik \\
	Springer international publishing\\
	ISBN: 978-3-319-15188-5

	\bibitem{Ethanol_Measurement}
	\textit{“Evaluation of Ethanol Measuring Tehcniques”}\\
	S. Hennessey ; K. Payne\\
	\textit{Worcester Polytechnic Institute}, 2015\\

	\bibitem{Density_Measurement}
	\textit{"Density Measurement using modern oscillating transducers"}\\
	H. Stabinger\\
	\textit{South Yorkshire Trading Standards Unit}, 1994

	\bibitem{Nondestructive_Measurement}
	\textit{\textbf{Nondestructive Evaluation of Food Quality Theory and Practice}}\\
	S. N. Jha, et al.\\
	Springer Science \& Business Media\\
	ISBN 978-3-642-15795-0

	\bibitem{NIR_For_Spices}
	\textit{"Comparison of Near-and Mid-Infrared Spectroscopy for Herb and Spice Authenticity Analysis"}\\
	K. Lawson-Wood ; I. Robertson ; U. K. Seer Green\\
	\textit{PerkinElmer, Inc.}, 2016

	\bibitem{NIR_Spectroscopy_Ethanol}
	\textit{"Noninvasive Method for Monitoring Ethanol in Fermentation Processes Using Fiber-optic Near-Infrared Spectroscopy"}\\
	A. G. Cavinato ; D. M. Mayes ; Z. Ge ; J. B. Callis\\
	\textit{Analytical chemistry 62, no. 18: 1977-1982}, 1990

	\bibitem{Gas_Chromatography_beer}
	\textit{"A rapid method for determination of ethanol in alcoholic beverages using capillary gas chromatography"}\\
	M. L. Wang; Y. M. Choong ; N. W. Su ; M. H. Lee\\
	\textit{Journal of Food and Drug Analysis, 11(2)}, 2003

	\bibitem{Ehtnaol_adsorption}
	\textit{"Adsorption of ethanol on Si (1 0 0) from first principles calculations"}\\
	P. L. Silvestrelli\\
	\textit{Surface science, 552(1-3), 17-26}, 2004

	\bibitem{Ethanol_in_Wine_GC}
	\textit{"Quantitative determination of ethanol in wine by gas chromatography"}\\
	Stackler, B., $\&$ Christensen, E. N. (1974)\\
	\textit{American Journal of Enology and Viticulture, 25(4), 202-207.}
	
	\bibitem{Hydrometer_Pic}
	\textit{"Racking the Hard Cider"}\\
	\textit{http://www.windward.org/notes/notes70/andrew7010.htm}\\
	Accessed on 26.3.2018

	\bibitem{HC-SR04}
	\textit{"Product User’s Manual – HCSR04 Ultrasonic Sensor"}\\
	Cytron Technologies Sdn. Bhd.
	
	\bibitem{RasPi0W}
	\textit{Pimoroni Tech Store}
	\textit{"https://shop.pimoroni.com/products/raspberry-pi-zero-w"}
	
	\bibitem{DHT11}
	\textit{Temperature and humidity module DHT11 Product Manual
		www.aosong.}\\
	Aosong(Guangzhou) Electronics Co.,Ltd
	
\end{thebibliography}

\ctutemplate{specification.as.chapter}

\end{document}
