
\catcode`<=13
\def<#1>{\hbox{$\langle$\it#1\/$\rangle$}}
\def\OPmac{\hbox{OPmac}}


\chap Části dokumentu

Tento dokument nemůže sloužit jako návod k~použití \TeX{}u a \csplain{}u.
Domnívám se ale, že metodou analogie je schopen i \TeX{}ový nováček vytvořit obvyklý
dokument. Doporučuji mu prostudovat stručný text~\cite[prvni] a uživatelskou 
dokumentaci k~makru \OPmac~"opmac-u.pdf"~\cite[opmac].
Na jednotlivé odstavce této dokumentace k~\OPmac{} budu v~této kapitole odkazovat.
Nově je též k dispozici text o základech plain\TeX{}u~\cite[tpp].


\sec Kapitoly, sekce, podsekce, přílohy

Dokument lze členit na kapitoly, sekce, podsekce a přílohy.
Používají se k~tomu příkazy vysvětlené v~dokumentaci k~\OPmac{} v~sekci třetí,
Kromě příkazů z~\OPmac{} makro \ctustyle{} přidává příkaz "\app".
Takže seznam příkazů pro vymezení základní struktury dokumentu vypadá takto:

\begtt
\chap Nadpis kapitoly <ukončený koncem řádku>
\sec Nadpis sekce <ukončený koncem řádku>
\secc Nadpis podsekce <ukončený koncem řádku>
\app Nadpis přílohy <ukončený koncem řádku>
\endtt
Upozornění: Až do verze OPmac Mar. 2016 bylo nutné za konec řádku u titulků
kapitol, sekcí atd. umístit ještě prázdný řádek. Doporučuji tedy ukončovat
tyto titulky prázdným řádkem, aby makra fungovala i ve starších verzích
OPmac.

Přílohy uvozené příkazem "\app" se chovají stejně jako kapitoly, jen nejsou
číslovány, ale jsou k~nim automaticky vzestupně přiřazena písmena A, B,
C\dots{} Také přílohy mohou být členěny na sekce a podsekce.


\sec Povinné části závěrečné práce

\secc Rozvržení dokumentu

Je doporučeno číslované kapitoly zahájit úvodem, pak další kapitoly podle
potřeby práce a poslední kapitola by měla být označena jako závěr. 
Následně musí být zařazen seznam literatury
(nečíslovaná kapitola) a pak povinná příloha A se zadáním práce 
(u~bakalářské a diplomové práce) a následně případné další přílohy. Mezi těmito
dalšími přílohami se velmi doporučuje zařadit seznam zkratek a symbolů, jako
zde v příloze~\ref[zkratky].


\secc Literatura

Každá studentská závěrečná práce musí obsahovat seznam použitých zdrojů. 
Toto je jediná nečíslovaná kapitola umístěná na konci textu práce ale před
přílohami. \ctustyle{} nabízí příkaz "\bibchap",
který je pořeba použít místo příkazu "\chap". Příkaz vytvoří záznam pro
obsah, vytiskne slovo Literatura (nebo References v anglicky psané práci).
Toto slovo se generuje automaticky, příkaz zapisujte bez parametru.

Způsob tvorby seznamu literatury a způsob citace na literaturu v~textu je
podrobněji vyložen v~sekci~\ref[citlit] nebo~\ref[citiso].


\secc Zadání práce

Každá bakalářská nebo diplomová závěrečná práce musí obsahovat úplný text zadání práce. 
Doporučuji jej zařadit do přílohy~\ref[zadani]~(jako v~tomto dokumentu)
a označit ji názvem \uv{Zadání práce} (anglicky Specification).
Někdy je v požadavcích uvedeno vložit zadání práce hned za titulní stranu.
Ačkoli to z typografického pohledu nedoporučuji, \ctustyle{} to umožňuje.
Podrobnější komentář k této problematice je v~sekci~\ref[ozadani]


\sec Obrázky, tabulky, listingy a další

\secc Obrázky

Obrázky ve formátu JPG, PNG (bitmapové) nebo PDF (vektorové i bitmapové) je
možné vložit příkazem "\inspic". Viz sekci 12 v~dokumentaci k~\OPmac. Pod
obrázek je nutné vložit popisek příkazem "\caption/f", viz sekci 4
v~dokumentaci k~\OPmac. \ctustyle{} navíc definuje příkaz "\cinspic", který
umístí obrázek doprostřed. Například:

\begtt
\medskip
\picw=5cm \cinspic ctulogo.pdf
\caption/f Ukázka vložení obrázku na střed, což je asi nejobvyklejší.
\medskip
\endtt
%
vytvoří:

\medskip  \clabel[logo]{Ukázka vložení obrázku na střed}
\picw=5cm \cinspic ctulogo.pdf
\caption/f Ukázka vložení obrázku na střed, což je asi nejobvyklejší.
\medskip

Makro "\cinspic" vyžaduje jméno souboru s příponou ukončené mezerou podobně
jako příkaz "\input". 

Pokud se obrázek vyskytuje dole na stránce tak, že stranu přeplní, nastávají
problémy se stránkovým zlomem. Proto je vhodné obrázky vložit i s~popiskem
do dvojice příkazů "\midinsert" a "\endinsert". V~takovém případě obrázek
implicitně zůstane, kde je, ale při potížích odpluje na začátek následující
stránky:

\begtt
\midinsert
\picw=5cm \cinspic ctulogo.pdf
\caption/f Ukázka vložení obrázku na střed, což je asi nejobvyklejší.
\endinsert
\endtt

Místo příkazu "\midinsert" můžete použít "\topinsert". V~takovém případě
obrázek odpluje na horní část stránky vždy. Raději má vršek aktuální
stránky, ale když to není možné, odpluje na stránku následující.


\secc Tabulky

Pro tabulky použijte příkaz "\ctable{<deklarace>}{<data>}", 
který je odvozen z~příkazu "\table"
dokumentovaného v~\OPmac{} v~sekci~11. \ctustyle{} definuje "\ctable"
tak, že tabulku navíc podkládá modrým pozadím (což je součástí typografického
návrhu šablony) a centruje ji.
Například:

\begtt
...vytvoří tabulku~\ref[absolventiFEL].

\midinsert \clabel[absolventiFEL]{Počet absolventů FEL ČVUT}
\ctable{lrrrrr}{
 \hfil number of       & 2007 & 2008 & 2009 & 2010 & 2011 \crl \tskip4pt
 students Bc. and Mgr. & 6313 & 5913 & 5951 & 5188 & 4737 \cr
 graduate Bc. and Mgr. & 1195 & 1489 & 1379 & 1160 & 1260 \cr
 students Ph.D.        &  457 &  468 &  366 &  395 &  434 \cr
 graduate Ph.D.        &   65 &   60 &   55 &   54 &   51 \cr
}
\caption/t Počet absolventů FEL ČVUT. Tabulka je převzata z~\cite[zyka].
\endinsert
\endtt
%
vytvoří tabulku~\ref[absolventiFEL].

\midinsert \clabel[absolventiFEL]{Počet absolventů FEL}
  \ctable{lrrrrr}{
    \hfil number of       & 2007 & 2008 & 2009 & 2010 & 2011 \crl \tskip4pt
    students Bc. and Mgr. & 6313 & 5913 & 5951 & 5188 & 4737 \cr
    graduate Bc. and Mgr. & 1195 & 1489 & 1379 & 1160 & 1260 \cr
    students Ph.D.        &  457 &  468 &  366 &  395 &  434 \cr
    graduate Ph.D.        &   65 &   60 &   55 &   54 &   51 \cr
}
\caption/t Počet absolventů FEL ČVUT. Tabulka je převzata z~\cite[zyka].
\endinsert

Doporučuji nerámovat tabulky do dalších rámečků, ale využít toho, že tabulka
je automaticky ohraničena modrým podkladem. Je vhodné pouze vložit linku mezi
záhlaví a údaje v~tabulce (viz příkaz "\crl" v~ukázce).

Tabulky (podobně jako obrázky) je vhodné zapouzdřit do dvojice 
příkazů "\midinset" a "\endinsert" nebo "\topinsert" a "\endinsert".


\secc Automaticky číslované objekty

Jak je možné si všimnout, \ctustyle{} automaticky čísluje kapitoly, sekce,
podsekce, dodatky, tabulky, obrázky a pokud uživatel použije "\eqmark", 
očísluje i rovnice. O~tomto číslování a o~odkazech na tato čísla
v~textu pojednává odstavec 4 v~dokumentaci k~\OPmac. Zde jen stručně uvádím,
že číslované objekty je potřeba označit interním lejblíkem příkazem "\label"
a pak je možné na ně odkazovat příkazem "\ref". Existuje ještě možnost
odkazovat na stránku příkazem "\pgref" a na literaturu příkazem "\cite".

Kapitoly se číslují od jedné v~celém dokumentu, sekce se číslují druhým
číslem v~pořadí od jedné v~každé kapitole a podsekce se číslují
třetím číslem od jedné v~každé sekci. Hlubší zanoření (podpodsekce) není
podporováno a není pro studentské práce doporučeno.

Tabulky se číslují od jedné v~každé kapitole a obrázky (nezávisle na
tabulkách) taky. Rovněž rovnice se číslují od jedné v~každé kapitole.
\ctustyle{} volí kompromis mezi krátkým číslováním (Tabulka 27) a dlouhým
číslováním (Tabulka 2.4.6). První extrém nedává představu o kapitole, ve
které je tabulka umístěna, a druhý extrém se čtenáři obtížně pamatuje.

\ctustyle{} definuje kromě příkazu "\label" ještě příkaz
"\clabel[<lejblík>]{<text>}", který funguje jako "\label[<lejblík>]", ale
navíc vloží takto označenou tabulku nebo obrázek do seznamu tabulek nebo
obrázků. Tyto seznamy se vygenerují hned za obsahem dokumentu. Pozor:
není-li tabulka nebo obrázek označen pomocí "\clabel", v~příslušném seznamu
se neobjeví. Někoho může napadnout otázka, proč má psát "<text>" dvakrát:
jednou pro seznam obrázků či tabulek v~příkazu "\clabel" a jednou pod
obrázek v~příkazu "\caption". Je to proto, že ty texty se mohou lišit.
Typicky v~obsahu budou stručnější. Ukázka použití "\clabel" je u~výpisu kódu
k~tabulce~\ref[absolventiFEL].

\ctustyle{} umožňuje použít automaticky číslované odstavce. Je připraveno pět
nezávislých čítačů označených A, B, C, D a E, každý z nich začíná 
v každé kapitole číslovat od jedné. Makro "\numberedpar<čítač>{<slovo>}"
zahájí číslovaný odstavec ve tvaru "<slovo> <číslo kapitoly>.<hodnota čítače>".
Následující příklad deklaruje věty a důsledky číslované společnou řadou
čísel a dále nezávisle číslované definice a příklady.

\begtt
\def\veta     {\numberedpar A{Věta}} 
\def\dusledek {\numberedpar A{Důsledek}} 
\def\definice {\numberedpar B{Definice}} 
\def\priklad  {\numberedpar C{Příklad}} 
\endtt

Po této deklaraci můžete psát "\definice Nechť $M$ je naprázdná ..."
a objeví se odstavec zahájený takto:

\def\definice {\numberedpar B{Definice}} 
\definice Nechť $M$ je neprázdná \dots

Další definice v této kapitole bude mít číslo 2.2, další 2.3 atd. K tomu
mohou být přidány věty a důsledky číslované 2.1, 2.2, atd. Konečně i
příklady v této kapitole budou číslovány 2.1, 2.2, atd. Před takto označené
odstavce lze psát "\label[<lejblík>]" a dá se pak na ně odkazovat pomocí
"\ref[<lejblík>]" a "\pgref[<lejblík>]", tedy odkazování je stejné jako u
všech ostatních automaticky číslovaných objektů.

\secc Listingy, výpisy kódů

Pro listingy, tj. výpisy kódu, použijte dvojici příkazů "\begtt" a "\endtt",
jak o~tom píše dokumentace k~\OPmac{} v~sekci~10. \ctustyle{} definuje
"\tthook" tak, aby byly listingy podbarveny světle modrou barvou, což je
součást grafického stylu.

Listingy se lámou do více stránek a jsou tištěny strojopisem, aby to
navodilo atmosféru pohledu do textového programátorského editoru, který
rovněž používá písmo s~pevnou šířkou všech znaků. Pravda, atmosféru to
nevytvoří dokonalou, protože textové editory dnes navíc používají prostředky pro
zvýraznění některých slov (klíčových slov programovacího jazyka atd.).
Chcete-li tedy navodit dokonalou atmosféru, uložte
si zobrazení svého textového editoru jako obrázek a do dokumentu vložte
obrázek. Nebo můžete experimentovat s OPmac triky 0124 a 0125.%
\fnote{\url{http://petr.olsak.net/opmac-tricks.html\#hisyntax}} 
Ovšem strohé listingy jen pomocí "\begtt" a "\endtt" jsou velmi doporučené,
protože modrý podklad graficky ladí s~celkovým návrhem \ctustyle{}.

Pokud chcete přímo v~odstavci uvádět kusy kódů, obalte je do dvojice 
znaků {\tt\dprime...\dprime}. Je možné tedy psát třeba toto:

\begtt
Chcete-li zdůraznit slovo, použijte {\em kurzívu}, do které
přepnete příkazem "\em", tedy "{\em zvýrazněné slovo}". 
\endtt

Tyto kusy kódu budou uvnitř odstavce tištěny strojopisem a nebudou podléhat
řádkovému zlomu. Bohužel dvojice znaků {\tt\dprime...\dprime} je možné použít jen
uvnitř \uv{obyčejného} odstavce, nikdy nefungují uvnitř parametrů jiných
příkazů (nadpisy kapitol, obsahy tabulek, atd.). V~takových místech musíte
do strojopisu přepnout explicitně pomocí "{\tt text}" a pohlídat si sazbu
\TeX{}ovsky citlivých znaků. Místo backslashe je možné psát "\bslash" a místo
procenta "\percent".  


\secc Poznámky pod čarou

Pro poznámky pod čarou používejte "\fnote{<text>}" jak je popsáno 
v~sekci~14 dokumentace k~\OPmac. Vytvoří to poznámku\fnote{Jako je tato.}. 
Poznámky pod čarou jsou číslovány na každé stránce od jedné. Doporučuji 
s~takovými poznámkami šetřit. 

Poznámky na okraji "\mnote", o~kterých také
hovoří dokumentace k~\OPmac, nejsou při použití \ctustyle{} doporučeny.


\secc Zvýrazňování textů

Základní text je psán antikvou (písmem Latin Modern odvozeným z~Computer
Modern). Chcete-li zdůraznit slovo, použijte {\em kurzívu}, do které
přepnete příkazem "\em", tedy "{\em zvýrazněné slovo}". 
Je to obvyklý způsob zdůrazňování, který je typograficky vhodný, protože
netrčí z~textu, ale je viditelný při čtení.

Pokud chcete zdůraznit něco, aby to bylo {\bf vidět z~dálky}, použijte
přepínač "\bf", který při použití \ctustyle{} přepíná do tučného fontu bez serifů (tj.
bez patek). Tedy "{\bf takto}". V~tomto fontu jsou řešeny i nadpisy.

Vyznačování podtrháním textu nebo prostrkáním nedoporučuji.


\label[uvozovky]
\secc Uvozovky, pomlčky, nezlomitelné mezery

{\bf České uvozovky} vypadají \uv{takto}, {\bf anglické} ``takto''. V~závislosti na
jazyce použijte správné uvozovky. Můžete je napsat přímo v~textovém editoru
(v~UTF-8 kódování), nebo \TeX{}ovsky to uděláte "\uv{takto}" pro češtinu a
"``takto''" pro angličtinu.

{\bf Pomlčky} v~typografii jsou dvě. 
\begitems
* Střední pomlčka: -- 
(používá se bez mezer kolem ve významu \uv{až} nebo s~mezerami 
jako pomlka ve větě). 
* Dlouhá pomlčka: --- 
(používá se v~anglickém textu). 
\enditems

Můžete tyto znaky napsat přímo v~editoru
v~UTF-8 kódování nebo \TeX{}ovsky: "--" (střední pomlčka), "---" (dlouhá
pomlčka). Čtenář vašeho textu vám strhne nemilosrdně body, pokud ve významu
pomlčky použijete spojovník. Vypadá takto: \uv{-} a promění se na něj
singl znak \clqq"-"\crqq{} ve zdrojovém textu.

{\bf Nezlomitelná mezera} je mezislovní mezera, ve které nedojde k~zalomení
do řádků. V~\TeX{}ových zdrojových textech se typicky tato mezera značí
vlnkou \clqq"~"\crqq. Existuje program 
"vlna"\urlnote{ftp://math.feld.cvut.cz/olsak/vlna/}, 
který dokáže
zaměnit normální mezery za tyto vlnky ve zdrojovém textu za všemi výskyty
neslabičných předložek, kam skutečně patří v~češtině i slovenštině
nezlomitelná mezera. O~to se tedy uživatel při psaní textu nemusí starat,
jen při závěrečných korekturách použije program "vlna" na všechny vstupní
soubory se zdrojový textem a spustí \TeX{} znovu. Program "vlna" ovšem nedává
vlnky před čísla citací a referencí a na mnoho míst, kam podle zvyklostí
v~sazbě taky patří. To si musí uživatel pohlídat sám.


\secc Odkazy do internetu.

Do internetu by se nemělo odkazovat přímo v textu, ale 
pomocí poznámky pod čarou "\fnote".
Aby se stalo URL klikatelné a bylo vytištěno správně strojopisem, je nutno
je vložit do parametru příkazu "\url", tedy třeba
"\url{http://petr.olsak.net}" vytvoří \url{http://petr.olsak.net}.
Ovšem navíc je potřeba tento text poslat do poznámky pod čarou. \ctustyle{}
definuje zkratku "\urlnote{<URL text>}", která je totožná 
s~"\fnote{\url{<URL text>}}". Takže text z předchozího odstavce byl napsán
takto:

\begtt
Existuje program "vlna"\urlnote{ftp://math.feld.cvut.cz/olsak/vlna/},
který dokáže...
\endtt

\secc Seznamy

Tvorba seznamů s~odrážkami je popsaná v~sekci 5 v~dokumentaci k~\OPmac{} 
(příkazy "\begitems" a "\enditems"). Implicitní odrážku v~seznamu 
definuje \ctustyle{}
jako modrý poněkud zaoblený čtvereček. Podívejte se, jak to vypadá, do 
sekce~\ref[uvozovky] do místa, kde se mluví o~pomlčkách.
Pokud chcete použít seznam v~seznamu,
pro vnitřní seznam použijte "\style x", což vytvoří poněkud menší modré
tečky. Pro číslované seznamy použijte "\style n".


\secc Slovníček zkratek

Tato možnost je zařazena do \ctustyle{} od verze May~2014.
Je možné si například do souboru "glosdata.tex" připravit následující obsah

\begtt
\glos {ČVUT}   {České vysoké učení technické v Praze}                          
\glos {FEL}    {Fakulta elektrotechnická ČVUT}
\glos {UK}     {Univerzita Karlova}
\glos {MFF}    {Matematicko-fyzikální fakulta UK}
\endtt
%
a zařadit jej do dokumentu na jeho začátek (nejlépe před "\makefront")
pomocí "\input glosdata". Tím se ještě nic nestane. Nyní ale můžete někam do
dokumentu napsat třeba

\begtt
\app Slovníček\par \makeglos
\endtt
%
a v uvedeném místě se objeví slovníček sestavený z "glosdata.tex" a
uspořádaný podle abecedy, třebaže glosdata uspořádána dle abecedy nejsou.
Chcete-li vypnout abecední řazení, pište na začátek dokumentu 
"\let\dosorting=\relax".\rfc{zkusím něco doplnit}.

Objeví-li se zkratka ze slovníčku někde v dokumentu, můžete ji označit pomocí
příkazu "\glref", například "\glref{ČVUT}", a v tomto místě se vytvoří
hypertextový odkaz do slovníčku.

Je též možné místo "\glref" použít jiné makro "\glosref{<zkratka>}{<význam>}". 
To využijete tehdy, pokud {\em nechcete} mít
souhrnný soubor "glosdata.tex", ale chcete významy jednotlivých zkratek
zapsat až v místě jejich výskytu. To má ale jistá omezení. 
Zatímco zkratka označená "\glref" se v dokumentu může
vyskytovat vícekrát, její význam musí být deklarován právě jednou
a deklarace významu všech zkratek musejí předcházet místu, kde je slovníček vytištěn
pomocí "\makeglos".

Chcete-li slovníček sestávající z více podsekcí, je nutné jej sestavit
manuálně, jako například v příloze~\ref[zkratky] tohoto dokumentu.

\secc Rejstřík

Rejstřík se u~studentských závěrečných prací nevyžaduje. Nic ale nebrání jej
vytvořit a postupovat přitom podle odstavce 7 v~dokumentaci k~\OPmac.


\label[citlit]
\sec Citace na literaturu, poznámky k~bib\TeX{}u

Odkazy v textu vytváříme příkazem "\cite[<lejblík>]". Lejblíků může být v
hranaté závorce více a jsou odděleny čárkou.
Podrobněji je tato problematika popsána v~manuálu k~\OPmac{}~\cite[opmac] v~sekci 15.

Seznam použité literatury má být řazen podle pořadí odkazů na citace v~textu.
Toto je součástí zadání, viz přílohu~\ref[zadani]. Osobně se mi to jeví jako nerozumné
rozhodnutí, ale zadání je třeba ctít. Použijete-li Bib\TeX{}, je doporučeno
použít Bib\TeX{}ový styl "plain", postupovat dle dokumentace k
OPmac a vytvořit přílohu se seznamem literatury takto:

\begtt
\bibchap
\usebbl/c mybase
\endtt

\iffalse
Na rozdíl od doporučení v~\cite[zyka] není použit 
Biber\urlnote{http://biblatex-biber.sourceforge.net/}, 
ale starý dobrý Bib\TeX{}\urlnote{http://www.bibtex.org/}, 
protože Biber je příliš komplikovaný a úzce
svázaný s~\LaTeX{}ovým stylem 
Bib\LaTeX{}\urlnote{http://ftp.cstug.cz/pub/tex/CTAN/help/Catalogue/entries/biblatex.html}. 
\fi

Asi nejpohodlnější možností k vytvoření seznamu literatury 
je přímé čtení databázového souboru ".bib" makry \TeX{}u bez použití
bib\TeX{}u. Toto je nová možnost (od dubna 2014). Stačí si připravit ".bib"
soubor s odpovídajícími údaji (například "mybase.bib"), přidat do záhlaví
dokumentu příkaz "\input opmac-bib" a do místa, kam má být vložen seznam 
literatury, napsat

\begtt
\bibchap
\usebib/c (simple) mybase
\endtt

Můžete vyjít z existujícího souboru "mybase.bib" a přidat si tam další položky. Je
tedy možné postupovat metodou analogie. Místo stylu "simple" je možné použít
též styl "iso690".

Upozornění: přímé čtení ".bib" souboru vyžaduje OPmac ve verzi aspoň
Apr.~2014 a k~němu soubory "opmac-bib.tex" a "opmac-bib-simple.tex",
které jsou dostupné na\urlnote{http://petr.olsak.net/opmac.html}.
Při přechodu na tuto novější verzi OPmac si laskavě smažte pomocné ".ref"
soubory, protože jejich přítomnost v aktuálním adresáři může způsobit chybu
v překladu.


\label[citiso]
\sec Citace na literaturu podle normy ČSN ISO 690

Jak si pozorný čtenář mohl všimnout, předchozí sekce popisuje tvorbu
seznamu literarury, který není v souladu s normou~\cite[csn690]. Nevím 
o~Bib\TeX{}ovém stylu, který by respektoval tuto normu z~roku 2007. Navíc
tato norma doporučuje vkládat do záznamů česká slova,
(například \uv{dostupné z}, \uv{vydání}, spojku \uv{a} mezi autory), 
což tedy není
použitelné v~případě, že je práce napsaná v angličtině. 

Pokud přesto vedoucí práce trvá na dodržení normy, doporučuji následující
postup. Seznam literatury si vygenerujte na\urlnote{http://www.citace.com}.
Od dubna 2014 nabízejí na těchto stránkách na můj podnět export do \TeX{}u.
Uložte tento export do "bbl" souboru a případně pozměňte české fráze,
píšete-li práci v angličtině\fnote
{Frázi \uv{dostupné z:} $\to$ \uv{available from}, spojku \uv{a} (bez čárky před) $\to$
\uv{and} (s čárkou před) atd.}.
Takto připravený seznam literatury můžete
načíst do dokumentu příkazem "\usebbl/c <jméno souboru>". Do dokumentu budou
zařazeny jen citované publikace a v pořadí podle citování.

Další možností je přímé čtení ".bib" souboru příkazem "\usebib" se stylem
"iso690".

\label[ozadani]
\sec Jak vložit přílohu A

Předpisy požadují vložit přílohu \uv{Zadání práce} (v případě bakalářky a
diplomky) do jednoho svázaného
výtisku jako originál (s podpisy a razítkem) a do dalších výtisků resp. do
PDF vložit kopii tohoto formuláře. Doporučuji tedy tento forulář oskenovat a
oříznout mu bílé okraje co nejvíce to jde. 
To lze udělat například v programu Gimp. Tím
vznikne obrázek například "zadani.jpg". Přílohy pak zahajte takovým
kódem:

\begtt
\app Zadání práce

\picw=\hsize % obrázek na šířku sazby
\cinspic zadani.jpg
\nextoddpage

\app Další příloha
\endtt

Makro \ctustyle{} zahajuje vždy první přílohu (tj. přílohu A) na liché
(pravé) straně. Za konec poslední kapitoly může tedy kvůli tomu vložit
prázdnou stranu (neboli vakát). Na první straně přílohy se tedy objeví
nadpis: \uv{Příloha A / Zadání práce} graficky upravený podle \ctustyle{}.
Pod ním bude oskenovaný formulář se zadáním práce.
Pak příkazem "\nextoddpage" dáváte najevo, že další příloha bude začínat
znovu na pravé stránce, takže vlevo se vloží vakát. Tím pádem bude příloha~A
zaujímat při duplexním tisku kompletní jeden list papíru v~závěrečné práci.

Nyní stačí tento list papíru v jednom výtisku zaměnit za originální formulář a nechat
svázat. Originál tedy nebude mít nadpis přílohy~A v jednotném grafickém stylu, ale
nemůžeme chtít všechno. Kopie už budou mít jednotný grafický styl.

Pokud předpisy bezpodmínečně požadují vložit zadání práce jako druhý list
hned za titulní stranu, je možné využít deklarační příkaz "\specification",
který zařadíte mezi ostatní deklarační příkazy a napíšete tam:

\begtt
\specification {\picw=\hsize \cinspic zadani.jpg }
\endtt

Tato deklarace způsobí, že sken zadání (bez dalších textů) se umístí na
stranu třetí (strana druhá je typicky vakát za titulem, pokud není
deklarován "\pagetwo"). Strana čtvrtá se stane dalším vakátem a od strany páté
se zobrazí poděkování/prohlášení atd. Navíc na stranu třetí bude odkazovat
strukturovaný obsah PDF prohlížeče názvem Zadání, resp. Specification.

Máte-li dvě varianty zadání (např. ve dvou jazycích), je možné psát:

\begtt
\specification {\picw=\hsize \cinspic specifi-en.jpg 
                \vfil\break  \cinspic specifi-cz.jpg }
\endtt

V tomto příkladě se "specifi-en.jpg" zobrazí na straně třetí a "specifi-cz.jpg" na
straně čtvrté. A stejně jako prve od strany páté pokračuje 
poděkování/prohlášení atd.


\sec Pracovní verze dokumentu

Příkazem "\draft" vloženým před příkaz "\makefront" vznikne verze dokumentu
označená datem vzniku a slovem Draft na každé stránce. Je to tedy pracovní
(nefinální) verze.

V~pracovní verzi jsou dále červeně vypsány lejblíky, které jste do dokumentu
vložili pomocí "\label" nebo "\clabel". Jsou umístěny v~místě cíle odkazů.
Při přechodu do finální verze (odstraněním příkazu "\draft") samozřejmě
lejblíky zmizí.

Při tvorbě dokumentu lze využít příkazy "\rfc{<poznámka>}", 
které v sazbě neudělají nic. Je-li ale zapnutý
"\draft", pak se souhrnný seznam těchto "<poznámek>" vypíše na uplně poslední stranu
dokumentu a je zpětně prolinkován s místy, kde byly jednotlivé příkazy "\rfc"
použity. RFC je zkratka za request for correction. Inspirace:

\begtt
\rfc{Tady musím doplnit obrázek}
...
\rfc{Ověřit, zda hodnoty v tabulce jsou OK}
\endtt
 
Jakmile je aktivován "\draft", můžete příkazem "\linespacing=<násobek>"
určit řádkování větší než implicitní řádkování 1. Například
"\linespacing=1.7". Tím se mezi řádky ve výstupním PDF dokumentu 
objeví mezery, do kterých může korektor v~pracovní verzi dokumentu 
vpisovat své poznámky. Při každé změně "\linespacing" je třeba \TeX{}ovat
aspoň dvakrát, aby se srovnalo stránkování v obsahu.

Upozorňuji, že řádkování rozdílné od implicitního řádkování 1, je pouze pro
účely pracovních verzí. Finální verze dokumentu {\em musí} mít řádkování 1. 
Ignorování této zásady bude považováno za nedodržení oficiálního
stylu pro závěrečné práce na ČVUT. Proto taky \ctustyle{} při odstranění příkazu
"\draft" automaticky deaktivuje nastavení "\linespacing".

Velké mezerování mezi řádky bylo dříve
doporučováno pro psaní studentských závěrečných prací, ale všichni lidé, kteří
něco vědí o~typografii, se snaží toto desítky let staré nařízení (vyplývající
z~technologie mechanických psacích strojů a z~normy, podle které autor
odevzdával své rukopisy pořízené na takovém psacím stroji tiskárně) jednoznačně vypudit
jako něco, co nemá při dnešních možnostech pořizování dokumentů žádné
opodstatnění. Typografie je nástroj, kterým předáváme své myšlenky dalším
čtenářům a ten nástroj nesmí čtenáře rušit a unavovat ve čtení.
%
Zmíněná starodávná norma měla za úkol usnadnit tiskárenskému závodu
spočítat počet znaků knihy, které autor dodal v~rukopise, a na základě toho
určit cenu prací. Pokud je potřeba zjistit počet znaků v~současném dokumentu, 
můžete to udělat jednodušeji, například příkazem:

\begtt
pdftotext dokument.pdf - | wc -m
\endtt

Tento příkaz spočítá i znaky v~automaticky generovaném obsahu. Pokud toto
není žádoucí, je možné přepínačem "-f" programu "pdftotext" specifikovat, od
které stránky PDF dokumentu má začít číst. Je třeba tam uvést absolutní číslo
strany PDF dokumentu, nikoli čísla podle stránkových číslic.

Příkaz "\savetoner" umožní vypnout (provizorně při "\draft") modré podklady
pod listingy a tabulkami. Ve finálním dokumentu (po vypnutí "\draft") 
jsou podklady vždy podbarveny.

Příkaz "\blackwhite" přepne modrou barvu na šedou. Chcete-li tisknout
nakonec černobíle, je možná lepší použít tuto variantu dokumentu. Pro
finálně vygenerované PDF (které není určeno k tisku) ovšem doporučuji 
vrátit se k barvě.

Implicině se předpokládá tisk na duplexové tiskárně.
Příkaz "\onesideprinting" přepne záhlaví do formy vhodné pro jednostranný
tisk. 


\sec Teze práce

V případě disertační práce se vyžaduje vytvořit teze práce, tedy extrakt
práce ve formátu A5, který mívá pro titulní a informativní stránku
předepsaný formát a je omezen na cca 20 stran textu. Od verze \ctustyle{}
May~2014 je do balíčku zahrnut makrosoubor "ctustyle-ts.tex" a výchozí
příklad "example-ts.tex", pomocí kterého snadno vytvoříte teze vaší práce.
Soubor "example-ts.tex" si překopírujte do svého souboru a vyměňte
připravené údaje za údaje dle vaší práce. V tomto souboru je uvedeno, které
údaje jsou povinné, které nepovinné a pomocí kterých je možné přepsat
implicitní texty.

Implicitní texty jsou šity na teze {\em disertační} práce (a jsou připraveny
v~češtině a angličtině). Ovšem jejich předefinováním si můžete vytvořit teze
jakékoli práce.

Při psaní tezí práce je využit \ctustyle{}, takže vše funguje stejně, jak je
posáno zde. Rozdíl je jen v tom, že není vhodné text členit na kapitoly (jen
na sekce a podsekce) a pro automatické generování obahu je potřeba přímo použít
příkaz "\maketoc". Vše je uvedeno v~ukázce. 

Výsledný PDF soubor s tezí práce můžete zpracovat příkazem "pdfbook soubor.pdf"
a výsledek ve tvaru "soubor-book.pdf" vytisknout na duplexové tiskárně.
Dostanete svazeček listů A4, který se přehnutím v půli promění v knížtičku
se stránkami A5. Příkaz "pdfbook" je z balíčku "pdfjam", který se opírá o
"pdflatex". Je také možné \LaTeX{} obejít a udělat to přímo v plain\TeX{}u,
jak je popsáno v~\cite[tpp] v sekci~11.8.


\chap Poznámky k~typografii

Šablona \ctustyle{} řeší následující věci, které jsou na sobě víceméně 
nezávislé:

\begitems
* Strukturu dokumentu a vymezení jeho povinných částí.
* Způsob, jak vyznačovat jednotlivé části ve zdrojovém textu dokumentu.
* Vzhled výstupu, neboli typografii. To je obsahem této sekce.
\enditems

Vyšel jsem ze zadání v~dodatku~\ref[zadani]. Dále jsem čerpal  
z~{\em Grafického manuálu identity ČVUT}~\cite[grafman],
který určuje, jak mají vypadat tiskoviny naší univerzity.
V~tomto manuálu je doporučeno střídat kromě černé barvu 
\hbox{\locc\Blue Pantone 300~C (blankytně modrou)} jako výrazný znak tiskovin ČVUT.
Tomuto doporučení jsem vyhověl. Mým cílem bylo oživit typografii závěrečných
prací tak, aby se to s~radostí četlo i psalo. Je v~tom skryto trochu
hravosti a rozpustilosti, ale domnívám se, že jen v~takové míře, v~jaké není
narušen slavnostní ráz a důležitost studentské závěrečné práce.
Navržená šablona může být použita na všech fakultách ČVUT.

Domnívám se, že modrá barva
na monitoru působí dobře a při tisku na černobílé tiskárně se holt
promění v~šedou, ale to je pro prezentaci práce v~tištěné podobě dostačující.
Barevný tisk je navíc pro studenty stále dostupnější.

Předpokládám, že se uživatel nebude v~barvách omezovat, když bude vkládat do
dokumentu schémata a obrázky, ovšem jistou střídmost by měl dodržet.

Oranžová barva (doplňková k~modré) je jen 
\urllink[url:http://petr.olsak.net]{navigační}. Naznačuje čtenáři, že
je vyznačená oblast textu klikací. Tyto oranžové rámečky zcela zmizí při
tisku, protože vytištěný text už pochopitelně klikací není.

Nevyhověl jsem doporučení manuálu~\cite[grafman] v~případě fontů, protože
toto doporučení bylo v~rozporu s~požadavkem z~dodatku~\ref[zadani].
Tam se požaduje písmo Latin Modern, zatímco v~Grafickém manuálu se požaduje
písmo Times. Mé oči vnímají obě písma jako hodně okoukaná. Na novém a 
neokoukaném písmu se podílím v~jiném projektu, ale ten je
nyní teprve v~počátcích a není k~zařazení do šablony bohužel připraven. Rozhodl
jsem se zůstat u~Latin Modern i z~toho důvodu, že v~tomto písmu vypadají hodně 
dobře matematické vzorečky.

Předpokládám, že dokument bude tištěn na duplexové tiskárně, po svázání
listů budou liché stránky vpravo a sudé stránky vlevo. V~souvislosti s~tím
jsou nastaveny vnitřní okraje větší, protože předpokládám, že tyto okraje se
\uv{utopí} ve vazbě závěrečné práce. Konečně plovoucí záhlaví je navrženo
tak, že svými světle modrými tečkami z~vazby jakoby vychází na obě strany
k~vnějšímu okraji.

